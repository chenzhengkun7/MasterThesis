\chapter{Vibration Analysis & Feature Extraction for Deep Learning}

\chapterintrobox{Feature extraction is an important preprocessing step in the workflow of developing machine learning models and directly influences the model's performance. Therefore, this step should be carried out carefuly in order to extract meaningful features from the raw data. Vibration data carries very useful information about the system health, yet it needs a lot of preprocessing before being used as an input for a specific model. This chapter describes some of signal processing techniques used in traditional vibration analysis, but in this context they will be used as feature extractors for a neural network architecture.}

\section{Signal processing for vibration data}
Vibration data is vital for health assessment of a given system and carries out very useful information about its performance (refer to section \ref{sec:data-acquisition} for details about data acquisition), yet these information are usually hard to observe in its raw waveform. Signal processing techniques are used to convert the raw waveform from time domain into frequency or time–frequnecy domains.
Vibrations data 

\section{Fourier analysis}
Fourier analysis, also called harmonic analysis, of a periodic signal $x(t)$ is the decomposition of the series into summation of sinusoidal components, where each sinusoid has a specific amplitude and phase.

The Fourier transform (FT) of a signal $x(t)$ can be mathematically given by equation \ref{equation:fourier-transform}:

\begin{equation}
    x(w) = \int_{-\infty}^{\infty}x(t)e^{-jwt}dt
    \label{equation:fourier-transform}
\end{equation}

\section{Wavelet transform}
Wavelet transform is also a spectral analysis tool, like Fourier transform. The main difference is that Fourier transform decomposes the signal into sinusoidal components, but wavelet transform decomposes it into a set of oscillatory functions called wavelets. Unlike sinusoids, wavelets are localized in time, thus wavelet transform doesn't only provide information about the frequency present in a signal but also the time of their occurence. Wavelet transform is a much better solution than Fourier transform when studying non-linear non-stationary signals (i.e. its frequency components vary with time).

\subsection{Continuous wavelet transform}

\subsection{Discrete wavelet transform}


\section{Conclusion}