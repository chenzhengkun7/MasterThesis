\chapter*{Résumé}
\vspace{-2cm}
Ce mémoire a exploré les applications des nouvelles techniques émergentes de l'intelligence artificielle et de l'apprentissage profond (réseaux de neurones en particulier) pour la maintenance prédictive, le diagnostic et le pronostic. De nombreuses architectures neurales telles que les réseaux de neurones entièrement connectés, convolutifs et récurrents ont été développées et testées sur des bases de données publiques telles que le C-MAPSS de la NASA, les bases de données de Case Western Reserve University et FEMTO afin de diagnostiquer l'état de santé des équipements et/ou de prédire la durée de vie utile restante (RUL) avant panne. De nombreuses procédures de traitement des données et d'extraction de caractéristiques ont été utilisées en combinaison avec des techniques d'apprentissage profond telles que la réduction de la dimensionnalité (analyse en composantes principales) et le traitement du signal (analyses de Fourier et d'ondelettes) afin de créer des caractéristiques plus significatives et plus robustes à utiliser en entrée des architectures de réseaux neuronaux. Ce mémoire a également exploré l'utilisation potentielle de ces techniques dans la maintenance prédictive au sein des plateformes pétrolières pour la surveillance des équipements critiques des champs pétrolifères afin de réduire les temps d'arrêt et les coûts de maintenance imprévus.

Mots clés: maintenance prédictive, pronostique, apprentissage profond, réseaux de neurones

\vspace{-1cm}
{\let\clearpage\relax\chapter*{Abstract}}
\vspace{-2.5cm}
This thesis explored applications of the new emerging techniques of artificial intelligence and deep learning (neural networks in particular) for predictive maintenance, diagnostics and prognostics. Many neural architectures such as fully-connected, convolutional and recurrent neural networks were developed and tested on public datasets such as NASA C-MAPSS, Case Western Reserve University Bearings and FEMTO Bearings datasets to diagnose equipment health state and/or predict the remaining useful life (RUL) before breakdown. Many data processing and feature extraction procedures were used in combination with deep learning techniques such as dimensionality reduction (Principal Component Analysis) and signal processing (Fourier and Wavelet analyses) in order to create more meaningful and robust features to use as an input for neural networks architectures. This thesis also explored the potential use of these techniques in predictive maintenance within oil rigs for monitoring oilfield critical equipment in order to reduce unpredicted downtime and maintenance costs.


Keywords: predictive maintenance, prognostics, deep learning, neural networks
