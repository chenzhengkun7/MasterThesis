\chapter*{Résumé}
\vspace{-2cm}


Mots clés: pronostic, fiabilité, RUL, apprentissage automatique

{\let\clearpage\relax\chapter*{Abstract}}
\vspace{-2cm}
This thesis explored applications of the new emerging techniques of neural networks and deep learning for predictive maintenance, diagnostics and prognostics. Many neural architectures such as fully-connected, convolutional and recurrent neural networks were developed and tested on public datasets such as NASA C-MAPSS, CaseWestern Reserve University Bearings and FEMTO Bearings datasets to diagnose equipment health state and/or predict the remaining useful life (RUL) before breakdown. Many data processing and feature extraction procedures were used in junction with deep learning techniques such as dimensionality reduction (Principal Component Analysis) and signal processing (Fourier and Wavelet analyses) in order to create more meaningful and robust features to use within neural networks. This thesis also explored the potential use of these techniques in predictive maintenance within oil rigs for monitoring oilfield critical equipment in order to reduce unpredicted downtime and maintenance costs.

Keywords: predictive maintenance, prognostics, deep learning, neural networks, signal processing
