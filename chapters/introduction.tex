\chapter*{Introduction}
\addcontentsline{toc}{chapter}{Introduction}
\markboth{Introduction}{Introduction}

Toutes les machines en général, et mécaniques en particulier, sont sujettes à la dégradation de leur état et à la détérioration de leurs performances avec le temps, ce qui entraîne éventuellement leur défaillance. Ces défaillances peuvent avoir des impacts négatifs sur les aspects économiques, humains, environnementaux, …
La défaillance d’une machine est une propriété intrinsèque de ces systèmes (en raison de leurs propriétés physiques inhérentes), elle peut être — afin d’éviter des résultats négatifs — partiellement prévenue, retardée et même prévue, mais elle ne peut jamais être totalement empêchée ou arrêtée, ceci est principalement réalisé par la maintenance.

La maintenance est définie comme un ensemble d’activités destinées à maintenir ou à restaurer une unité utilitaire dans un état dans lequel elle peut remplir une fonction requise \cite{ISO2015}.

La façon dont la maintenance industrielle est effectuée a évolué avec les progrès de la technologie depuis le début de la révolution industrielle jusqu’à aujourd’hui. Sa forme la plus élémentaire est l’entretien correctif non planifié qui est effectué après l’apparition d’une panne, afin de rétablir une unité fonctionnelle dans un état où elle peut remplir une fonction requise \cite{ISO2015}. Avant la Seconde Guerre mondiale, la machinerie était relativement simple et la demande de production était modérée, de sorte qu’elle pouvait être maintenue après une panne. Après la guerre et avec la reconstruction de l’industrie, le marché est devenu plus compétitif et moins tolérant au temps d’arrêt, l’industrie s’est donc tournée vers la maintenance préventive (\acrlong{pm}: \acrshort{pm})\footnote{Les termes seront mentionnés en Français, traduits et abrégés (selon la littérature) en Anglais} qui est effectuée à des intervalles prédéterminés ou selon des critères prescrits afin de réduire la probabilité de défaillance d’une unité fonctionnelle \cite{ISO2015}.  \textit{La maintenance corrective et préventive sont toutes deux obsolètes}. %L’objectif est de remplacer la maintenance « post-mortem » et « aveugle » par la maintenance « juste à temps » \cite{Fu2004}. 
C’était donc une question de temps avant que l’idée de la maintenance prédictive ou conditionnelle (\acrlong{cbm}: \acrshort{cbm}) émerge, ce qui diffère des autres formes en basant le besoin de l'intervention sur l’état réel de la machine plutôt que sur un calendrier préétabli \cite{Kadry2013}. Il y a un désaccord dans la littérature sur cette classification (de la \acrshort{cbm} en tant que maintenance préventive ou une forme distincte) \cite{Shin2015}. La différence de taxonomie n’est pas d’un intérêt réel pour la discussion actuelle.

Un nouveau domaine qui est apparu récemment est celui de pronostics et de gestion de la santé (\acrlong{phm} : \acrshort{phm}). Le \acrshort{phm} s’est imposé comme une approche essentielle pour obtenir des avantages concurrentiels sur le marché mondial en améliorant la fiabilité, la maintenabilité, la sécurité et l’abordabilité. Comme le \acrshort{cbm}, le \acrshort{phm} est une discipline qui a émrgé dans l'industrie avec une présence modèste dans le domaine académique, en particulier dans les applications militaires \cite{Tinga2014} (e.g. le développement de l'avion de combat F-35 \cite{Brown2007}).
Les concepts et les composants du \acrshort{phm} ont été développés séparément dans de nombreux domaines tels que le génie mécanique, électrique et les sciences statistiques, sous des noms variés \cite{Tsui2015}. Alors que le \acrshort{cbm} se concentre sur la surveillance du système, le \acrshort{phm} est une approche plus intégrée qui vise à fournir des lignes directrices pour la gestion de la santé du système. Il s’agit donc d’une philosophie de gestion du cycle de vie (\acrlong{lcm}: \acrshort{lcm}) qui met l’accent sur la prévisibilité (c.-à-d. le pronostic) des défaillances et de la maintenance. Ceci est généralement réalisé par l’adoption d’une stratégie de surveillance, qui peut être une technique de \acrshort{cbm} \cite{Tinga2014}. Le but principal du pronostic (aussi \acrshort{cbm}) est l'estimation de la durée de vie utile restante (\acrlong{rul}: \acrshort{rul}) en étant donné l'état actuel de l'équipement et son profil opérationnel passé \cite{Jardine2006}.

Le \acrshort{phm} agit à un niveau un peu plus élevé que le \acrshort{cbm}, puisqu’il a l’ambition claire de permettre la gestion de la santé. Il s’agit d’une activité liée à \acrshort{lcm}, ce qui signifie qu’une approche est suivie pour optimiser toutes les activités (de maintenance) pendant le cycle de vie complet du matériel. Cela comprend le choix d’une politique de maintenance appropriée, la définition de l'échéancier d'intervention et la décision sur le moment où un matériel doit être mis hors service. \acrshort{cbm} n’offre pas ce soutien \acrshort{lcm} étendu. Le domaine \acrshort{phm} ne prescrit ni un concept de maintenance spécifique ni une stratégie de surveillance. Cependant, dans les études typiques de \acrshort{phm}, des politiques de \acrshort{cbm} ou d’autres politiques de maintenance sont adoptées et dans de nombreux cas, les techniques de surveillance d'état (\acrlong{cm}: \acrshort{cm}) sont appliquées \cite{Tinga2014}.

Comme mentionné précédemment, tous ces efforts pour développer des stratégies de maintenance ont été principalement dus aux coûts élevés des pertes de production et de la maintenance réactive dans des industries à une forte intensité capitalistique comme les projets Onshore et Offshore dans l'industrie pétrolière et gazière sont principalement des investissements, qui peuvent avoir de graves conséquences financières et environnementales si une défaillance catastrophique se produit. Par conséquent, une approche efficace de la gestion de la maintenance est essentielle à la poursuite de la production de manière sûre et fiable \cite{Telford2011}. Les organisations Offshore subissent en moyenne des impacts financiers de 38 millions de dollars par an en raison de temps d'arrêt imprévus (pour les organisations les moins performantes, 88 millions de dollars). Moins de 24 \% des opérateurs décrivent leur approche de maintenance comme étant prédictive et basée sur des données et des analyses. Plus des trois quarts adoptent une approche réactive ou basée sur le temps. Les opérateurs utilisant une approche prédictive basée sur des données connaissent en moyenne 36 \% de temps d'arrêt non planifié de moins que ceux qui utilisent une approche réactive. Cela peut se traduire, en moyenne, par une baisse de 34 millions de dollars par an du résultat net \cite{Eriksen2016}.

Dans ce mémoire, une approche de pronostic basée sur les données (\acrlong{dd}) sera introduite, présentant les différentes étapes nécessaires depuis l'acquisition des données jusqu'à l'estimation de \acrshort{rul}. En raison de la sensibilité commerciale et technique des données dans le domaine pétrolier (et toutes les industries avec une grande valeur ajoutée) et donc de l'impossibilité de les obtenir, les bases de données publiques (NASA Ames Data Repository \cite{NASAAmes}) qui sont habituellement utilisées pour évaluer les algorithmes de pronostic dans la littérature seront utilisées içi aussi. Une approche sera proposée pour transférer les connaissances obtenues à des applications concrètes dans l'industrie pétrolière.
\newline
Ce mémoire est divisé comme suit:
\begin{description}
	\item[Chapitre 01] Vers un Approche de Pronostic Data-Driven
	\item[Chapitre 02] Les Etapes d'un Approche Data-Driven
	\item[Chapitre 03] Introduction au Réseaux des Neurones  
	\item[Chapitre 04] Évaluation de l'État des Équipements par les Réseaux de Neurones
	\item[Chapitre 05] Diagnostic et Pronostic des Roulements
	\item[Conclusion] 
\end{description}




