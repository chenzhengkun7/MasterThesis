\chapter*{Introduction}
\addcontentsline{toc}{chapter}{Introduction}
\markboth{Introduction}{Introduction}

All machines in general, and mechanical machines in particular, are subject to degradation of their condition and deterioration of their performance over time, possibly leading to their failure. These failures can have negative impacts on economic, human, environmental aspects.
Machine failure is an intrinsic property of these systems (due to their inherent physical properties), it can be - in order to avoid negative results - partially prevented, delayed and even foreseen, but it can never be totally prevented or stopped, this is mainly achieved through maintenance.

Maintenance is defined as a set of activities intended to maintain or restore a utility unit in a state in which it can perform a required function \cite{ISO2015}.

The way in which industrial maintenance is performed has evolved with advances in technology from the beginning of the industrial revolution to the present day. Its most basic form is the unplanned corrective maintenance that is carried out after a failure has occurred, in order to restore a functional unit to a state where it can perform a required function\cite{ISO2015}. Prior to World War II, machinery was relatively simple and production demand was moderate, so that it could be maintained after a breakdown. After the war and with the rebuilding of the industry, the market became more competitive and less tolerant of downtime, so the industry turned to preventive maintenance (\acrlong{pm}: \acrshort{pm}) which is carried out at predetermined intervals or according to prescribed criteria to reduce the probability of failure of a functional unit \cite{ISO2015}. Corrective and preventive maintenance are both obsolete. The goal is to replace "post-mortem" and "blind" maintenance with "just-in-time" maintenance. 
It was therefore only a matter of time before the idea of predictive or condition-based maintenance (\acrlong{cbm}: \acrshort{cbm}) emerged, which differs from other forms by basing the need for intervention on the actual state of the machine rather than on a pre-set schedule\cite{Kadry2013}. There is disagreement in the literature on this classification (of \acrshort{cbm} as preventive maintenance or a separate form)\cite{Shin2015}. The difference in taxonomy is not of real interest to the current discussion.

A new field that has recently emerged is that of health prognosis and management (\acrlong{phm}: \acrshort{phm}). \acrshort{phm} has emerged as an essential approach to achieving competitive advantages in the global marketplace by improving reliability, maintainability, safety and affordability. Like \acrshort{cbm}, \acrshort{phm} is a discipline that has emerged in industry with a modest presence in the academic field, particularly in military applications\cite{Tinga2014}. (e.g. the development of the F-35 fighter plane \cite{Brown2007}).
The concepts and components of the \acrshort{phm} have been developed separately in many fields such as mechanical, electrical and statistical engineering, under various names\cite{Tsui2015}. While \acrshort{cbm} focuses on system monitoring, \acrshort{phm} is a more integrated approach that aims to provide guidelines for system health management. It is therefore a life cycle management philosophy (\acrlong{lcm}: \acrshort{lcm}) that emphasizes predictability (i.e., prognosis) of failures and maintenance. This is usually achieved through the adoption of a monitoring strategy, which can be a technique of \acrshort{cbm} \cite{Tinga2014}. The main goal of the prognosis (also \acrshort{cbm}) is the estimation of the remaining useful life (\acrlong{rul}: \acrshort{rul}) given the current state of the equipment and its historic operational profile\cite{Jardine2006}.

The \acrshort{phm} operates at a slightly higher level than the \acrshort{cbm}, since it has the clear ambition to enable health management. This is an activity related to \acrshort{lcm}, which means that an approach is followed to optimize all (maintenance) activities during the complete lifecycle of the equipment. This includes choosing an appropriate maintenance policy, defining the intervention schedule and deciding when a piece of equipment should be taken out of service. \acrshort{cbm} does not offer this extensive \acrshort{lcm} support. The \acrshort{phm} domain prescribes neither a specific maintenance concept nor a monitoring strategy. However, in typical \acrshort{phm} studies, \acrshort{cbm} or other maintenance policies are adopted and in many cases, condition monitoring techniques (\acrlong{cm}: \acrshort{cm}) are applied\cite{Tinga2014}.

As mentioned earlier, all these efforts to develop maintenance strategies have been mainly due to the high costs of production loss and reactive maintenance in capital-intensive industries such as Onshore and Offshore projects in the oil and gas industry are mainly investments, which can have serious financial and environmental consequences if a catastrophic failure occurs. Therefore, an effective approach to maintenance management is essential to continue production safely and reliably \cite{Telford2011}. Offshore organizations experience an average financial impact of \$38 million per year due to unplanned downtime (for the worst performing organizations, \$88 million). Less than 24\% of operators describe their maintenance approach as predictive and based on data and analysis. More than three-quarters take a reactive or time-based approach. Operators using a predictive data-based approach experience on average 36\% less unplanned downtime than those using a reactive approach. This can translate, on average, into a \$34 million per year decline in bottom line earnings\cite{Eriksen2016}.

In this thesis, a data-driven prognostic approach will be introduced, outlining the various steps required from data acquisition to the estimation of \acrshort{rul}. Due to the commercial and technical sensitivity of the data in the petroleum field (and all industries with high added value) and therefore the impossibility to obtain them, the public databases (NASA Ames Data Repository \cite{NASAAAmes}) that are usually used to evaluate prognostic algorithms in the literature will be used here as well. An approach will be proposed to transfer the knowledge obtained to concrete applications in the petroleum industry.
\newline
This thesis is divided as follows:
\begin{description}
	\item [Chapter 01] Towards a Data-Driven Prognostics Approach
	\item [Chapter 02] Steps of a Data-Driven Approach
	\item [Chapter 03] Introduction to Artificial Neural Networks
	\item [Chapter 04] Equipment Health Assessment using Artificial Neural Networks
	\item [Chapter 05] Bearings Faults Diagnostics and Prognostics
	\item [Conclusion] 
\end{description}

