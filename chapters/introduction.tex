\chapter*{Introduction}
\addcontentsline{toc}{chapter}{Introduction}
\markboth{Introduction}{Introduction}
\begin{comment}
All machines in general and mechanical ones in specific, are subject to degradation of their state and deterioration of their performance with the passage of time which results eventually in failure of these machines, these failures can have negative economic, human, environmental and other impacts. Machine failure is an intrinsic property of these systems (due to their inherent physical properties), it can be—in order to prevent negative outcomes—partially prevented, delayed and even predicted but it can never be totally avoided or stopped, this is mainly achieved through maintenance.

Maintenance is formally defined as: set of activities intended to keep a functional unit in, or to restore it to, a state in which it can perform a required function .

The way industrial maintenance is carried out has evolved with the advance of technology from the beginning of the industrial revolution through World War I and II until today. Its most basic form is unplanned corrective maintenance which is carried out after occurrence of a failure, or detection of a fault, in order to restore a functional unit to a state in which it can perform a required function \cite{ISO2015}.

Prior to World War II, the machinery was relatively simple and the demand for production was moderate so it was adequate to maintain after a breakdown. But after the war during the 50s with the rebuilding of industry, the marketplace became more competitive with lower tolerance for downtime and production losses, thus the industry shifted to preventive maintenance which is performed at predetermined intervals or according to prescribed criteria in order to reduce the probability of failure or the degradation of the functioning of a functional unit \cite{ISO2015}.

Both breakdown maintenance and scheduled maintenance are out-of-date, and new maintenance method that accords with failure development procedures must be studied and applied to the industrial areas. The goal is to replace “postmortem” and “blind” maintenance with “just in time” maintenance \cite{Fu2004}.

So it was a matter of time before the idea of Predictive Maintenance or Condition-Based Maintenance (CBM) emerged, which differs from corrective and preventive forms by basing maintenance need on the actual condition of the machine rather than on some preset schedule \cite{Kadry2013}.

There is a disagreement in the literature on this classification, some researchers put CBM under preventive maintenance (alongside time-based maintenance) while others consider it a distinct form (predictive maintenance) \cite{Shin2015}. The difference in taxonomy isn't of a real interest to the current discussion.

A new field that emerged recently is Prognostics and Health Management (PHM). PHM has emerged as an essential approach for achieving competitive advantages in the global market by improving reliability, maintainability, safety, and affordability. Concepts and components in PHM have been developed separately in many areas such as mechanical engineering, electrical engineering, and statistical science, under varied names \cite{Tsui2015}.

Whereas CBM focuses on the monitoring of the system, PHM is a more integrated approach that aims to provide guidelines for managing the health of the system. In that way, it is a philosophy to perform Life Cycle Management (LCM), with a strong focus on the predictability (i.e. prognostics) of failures and maintenance. This is generally achieved by adopting some monitoring strategy, which may be a CM technique \cite{Tinga2014}.

PHM is acting on a somewhat higher level than CBM, since it has a clear ambition to enable health management. The latter is an activity related to Life Cycle Management (LCM), which means that an approach is followed to optimize all (maintenance) activities during the complete life cycle of the asset. This includes the selection of an appropriate maintenance policy, defining the maintenance interval length and deciding on the moment an asset should be discarded. CBM doesn't provide that extensive LCM support. The PHM field prescribes neither a specific maintenance concept nor a monitoring strategy. However, in typical PHM studies, CBM or other maintenance policies are adopted and in many cases CM techniques are applied \cite{Tinga2014}.
\end{comment}

Toutes les machines en général, et mécaniques en particulier, sont sujettes à la dégradation de leur état et à la détérioration de leurs performances avec le temps, ce qui entraîne éventuellement leur défaillance. Ces défaillances peuvent avoir des impacts négatifs sur les aspects économiques, humains, environnementaux, …
La défaillance d’une machine est une propriété intrinsèque de ces systèmes (en raison de leurs propriétés physiques inhérentes), elle peut être — afin d’éviter des résultats négatifs — partiellement prévenue, retardée et même prévue, mais elle ne peut jamais être totalement empêchée ou arrêtée, ceci est principalement réalisé par la maintenance.

La maintenance est formellement définie comme un ensemble d’activités destinées à maintenir ou à restaurer une unité utilitaire dans un état dans lequel elle peut remplir une fonction requise \cite{ISO2015}.

La façon dont la maintenance industrielle est effectuée a évolué avec les progrès de la technologie depuis le début de la révolution industrielle jusqu’à aujourd’hui, en passant par la Première et la Seconde Guerre mondiale. Sa forme la plus élémentaire est l’entretien correctif non planifié qui est effectué après l’apparition d’une panne ou la détection d’un défaut, afin de rétablir une unité fonctionnelle dans un état où elle peut remplir une fonction requise \cite{ISO2015}. Avant la Seconde Guerre mondiale, la machinerie était relativement simple et la demande de production était modérée, de sorte qu’elle pouvait être maintenue après une panne. Après la guerre et avec la reconstruction de l’industrie, le marché est devenu plus compétitif et moins tolérant au temps d’arrêt, l’industrie s’est donc tournée vers la maintenance préventive qui est effectuée à des intervalles prédéterminés ou selon des critères prescrits afin de réduire la probabilité de défaillance ou la dégradation du fonctionnement d’une unité fonctionnelle \cite{ISO2015}.  La maintenance corrective et préventive sont toutes deux obsolètes. %L’objectif est de remplacer la maintenance « post-mortem » et « aveugle » par la maintenance « juste à temps » \cite{Fu2004}. 
C’était donc une question de temps avant que l’idée de la maintenance prédictive ou conditionnelle (Condition-Based Maintenance: CBM) émerge, ce qui diffère des autres formes en basant le besoin de l'entretien sur l’état réel de la machine plutôt que sur un calendrier préétabli \cite{Kadry2013}. Il y a un désaccord dans la littérature sur cette classification (de la CBM en tant que maintenance préventive ou une forme distincte) \cite{Shin2015}. La différence de taxonomie n’est pas d’un intérêt réel pour la discussion actuelle.

Un nouveau domaine qui est apparu récemment est celui de \acrlong{phm} (\acrshort{phm}). Le \acrshort{phm} s’est imposé comme une approche essentielle pour obtenir des avantages concurrentiels sur le marché mondial en améliorant la fiabilité, la facilité d’entretien, la sécurité et l’abordabilité. Les concepts et les composants du \acrshort{phm} ont été développés séparément dans de nombreux domaines tels que le génie mécanique, électrique et les sciences statistiques, sous des noms variés \cite{Tsui2015}. Alors que le CBM se concentre sur la surveillance du système, le \acrshort{phm} est une approche plus intégrée qui vise à fournir des lignes directrices pour la gestion de la santé du système. Il s’agit donc d’une philosophie de gestion du cycle de vie ou Life Cycle Management (LCM) qui met l’accent sur la prévisibilité (c.-à-d. le pronostic) des défaillances et de la maintenance. Ceci est généralement réalisé par l’adoption d’une stratégie de surveillance, qui peut être une technique de maintenance conditionnelle \cite{Tinga2014}.

Le \acrshort{phm} agit à un niveau un peu plus élevé que le CBM, puisqu’il a l’ambition claire de permettre la gestion de la santé. Il s’agit d’une activité liée à la gestion du cycle de vie (LCM), ce qui signifie qu’une approche est suivie pour optimiser toutes les activités (de maintenance) pendant le cycle de vie complet du bien. Cela comprend le choix d’une politique d’entretien appropriée, la définition de l'échéancier d’entretien et la décision sur le moment où un bien doit être mis au rebut. CBM n’offre pas ce soutien LCM étendu. Le domaine \acrshort{phm} ne prescrit ni un concept de maintenance spécifique ni une stratégie de surveillance. Cependant, dans les études typiques de \acrshort{phm}, des politiques de CBM ou d’autres politiques de maintenance sont adoptées et dans de nombreux cas, les techniques de CM sont appliquées \cite{Tinga2014}.


%--------------------------------------------------------------------------

On- and off-shore development projects in the oil and gas industry are predominantly capital-intensive investments, with the potential for serious financial and environmental consequences should a catastrophic failure occur. Therefore, an efficient as well as effective maintenance management approach is essential to the continuation of production in a safe and reliable manner \cite{Telford2011}.

Offshore oil and gas organizations experience on average \$38 million annually in financial impacts due to unplanned
downtime (For the worst performers \$88 million). Fewer than 24\% of operators describe their maintenance approach as a predictive one based on data and analytics. Over three-quarters either take a reactive or time-based approach. Operators using a predictive, data-based approach experience 36\% less unplanned downtime than those with a reactive approach. This can result in, on average, \$34 million dropping to the bottom line annually \cite{Eriksen2016}.

In this thesis, a data-driven prognostic approach will be introduced, presenting the different needed steps from data acquisition to remaining useful life estimation. Due to the commercial and technical sensitivity of oilfield data (and all industries in general) consequently the inability to obtain it, publicly-available databases (from NASA Prognostics Center and other institutes) which are usually used to assess prognostics algorithms in literature will be used to develop and assess the algorithms' performance. An approach will be proposed to transfer the knowledge obtained here to real-world applications in the oil industry.
\newline
\begin{description}
	\item[Chapitre 01] defintiion
\end{description}




